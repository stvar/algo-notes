% Algorithms Notes (C) by Stefan Vargyas
% 
% This file is part of Algorithm Notes.
% 
% Algorithms Notes is licensed under a
% Creative Commons Attribution 4.0 Unported
% License.
% 
% You should have received a copy of the
% license along with this work.  If not, see
% <http://creativecommons.org/licenses/by/4.0/>.

\documentclass[a4paper,9pt,leqno]{article}
\usepackage[hang]{footmisc}
\usepackage[fleqn]{amsmath}
\usepackage{listings}
\usepackage{relsize}
\usepackage{tocloft}
\usepackage{quoting}
\usepackage{amsfonts}
\usepackage{amssymb}
\usepackage{amsthm}
\usepackage{suffix}
\usepackage{ifpdf}
\usepackage{calc}
\usepackage{url}

\ifpdf
\pdfinfo{
/Title(A String Minimum Deletions Problem)
/Author(Stefan Vargyas, stvar@yahoo.com)
/Creator(LaTeX)
}
\fi

\newcommand{\vargyas}{\c{S}tefan Vargyas}

\renewcommand{\ttdefault}{pcr}
\renewcommand{\familydefault}{\sfdefault}

\addtolength{\textheight}{3.4cm}
\addtolength{\topmargin}{-2.1cm}
\addtolength{\textwidth}{3cm}
\addtolength{\evensidemargin}{-1.5cm}
\addtolength{\oddsidemargin}{-1.5cm}

\renewcommand{\=}{\protect\nobreakdash-\hspace{0pt}}
\renewcommand{\~}{\protect\nobreakdash--\hspace{0pt}}

\setlength{\parindent}{0pt}

\newcommand{\plusplus}{\textbf{\raisebox{1pt}{++}}}
\newcommand{\cplusplus}{C\plusplus}
\newcommand{\python}{Python}

\newcommand{\ie}{i.e.}
\newcommand{\code}[1]{{\tt{#1}}}

\newcommand{\textpar}{%
    \par\noindent\newline}

\makeatletter
% stev: alter the definition of key 'numbers':
\lst@Key{showskiplines}f[f]{%
    \lstKV@SetIf{#1}\lst@ifshowskiplines}
\let\lst@ifskiplines\iffalse
\newcommand\skipnumbering[1]{%
    \setcounter{lstnumber}{\numexpr#1-1\relax}%
    \let\lst@ifskiplines\iftrue
    \\\lst@PlaceNumber
    \let\lst@ifskiplines\iffalse
    \lst@ifshowskiplines...\fi
}
\def\lst@renderskiplinessymbol{{%
    \relsize{1}%
    \lst@skiplinessymbol
    \hspace{-.11em}}%
}
\def\lst@numberorsymbol{
    \lst@ifskiplines
    \makebox[\widthof{\lst@renderskiplinessymbol}+.11em][l]{%
        \lst@renderskiplinessymbol}%
    \else
    \thelstnumber
    \fi
}
\def\lst@defskiplinessymbol{%
    \textbf{.\hspace{.11em}.\hspace{.11em}.}}
\def\lst@skiplinessymbol{}
\lst@Key{skiplinessymbol}{%
    \lst@defskiplinessymbol}{%
    \def\lst@skiplinessymbol{#1}}
\lst@Key{numbers}{none}{%
    \let\lst@PlaceNumber\@empty
    \lstKV@SwitchCases{#1}{%
        none&\\%
        left&\def\lst@PlaceNumber{%
            \llap{%
                \normalfont
                \lst@numberstyle{\lst@numberorsymbol}%
                \kern\lst@numbersep}}\\%
        right&\def\lst@PlaceNumber{%
            \rlap{%
                \normalfont
                \kern\linewidth \kern\lst@numbersep
                \lst@numberstyle{\lst@numberorsymbol}}}%
    }{%
        \PackageError{Listings}{Numbers #1 unknown}\@ehc
    }%
}
\makeatother

% http://www.bollchen.de/blog/2011/04/good-looking-line-breaks-with-the-listings-package/
\newcommand{\shellprebreak}{%
    \raisebox{0ex}[0ex][0ex]{%
        \ensuremath{\hookleftarrow}}}
\newcommand{\shellpostbreak}{%
    \raisebox{0ex}[0ex][0ex]{%
        \ensuremath{\hookrightarrow}\hspace{2pt}}}
\newcommand{\srcprebreak}{%
    \shellprebreak}
\newcommand{\srcpostbreak}{%
    \shellpostbreak}

\newlength\shelllinewidth
\setlength\shelllinewidth{1.35\linewidth}

\newcommand{\shellskipamount}{%
    \baselineskip}
\newcommand{\shellinnerskipamount}{%
    2\parskip}
\newcommand{\shellafterskipamount}{%
    \parskip}
\newcommand{\shellafterskip}{%
    \addvspace{-\shellafterskipamount}}%
\newcommand{\srcskipamount}{%
    \shellskipamount}
\newcommand{\srcafterskipamount}{%
    \shellafterskipamount}
\newcommand{\srcafterskip}{%
    \addvspace{-\srcafterskipamount}}%

\lstdefinestyle{srclistingstyle}{%
    numbers=left,
    numbersep=1.3em,
    numberstyle=\tiny,
    showstringspaces=false,
    basicstyle=\smaller\ttfamily,
    breakindent=0pt,breakautoindent=false,
    breaklines=true,breakatwhitespace=true,
    linewidth=\shelllinewidth,lineskip=1.05pt,
    abovecaptionskip=0pt,
    belowcaptionskip=0pt,
    %!!!aboveskip=\srcskipamount,
    %!!!belowskip=\srcskipamount,
    postbreak=\srcpostbreak,
    prebreak=\srcprebreak,
    emphstyle=\textbf,
}
\lstdefinestyle{cpplistingstyle}{%
    style=srclistingstyle,
    language=[GNU]C++,
    tabsize=4,
    emph={
        alignas,
        alignof,
        auto,
        decltype,
        nullptr,
        constexpr,
        final,
        override,
        static_assert,
        noexcept,
        __func__,
        __FILE__,
        __LINE__,
        __VA_ARGS__,
    },
}
\lstnewenvironment{cpplisting}{%
    \lstset{style=cpplistingstyle}%
}{}

\lstdefinestyle{pylistingstyle}{%
    style=srclistingstyle,
    language=Python,
    tabsize=4,
}
\lstnewenvironment{pylisting}{%
    \lstset{style=pylistingstyle}%
}{}

\swapnumbers
\theoremstyle{plain}
\newtheorem{fact}{Fact}
\newtheorem{axioms}[fact]{Axioms}
\newtheorem{proposition}[fact]{Proposition}
\newtheorem{theorem}[fact]{Theorem}
\theoremstyle{definition}
\newtheorem{definition}[fact]{Definition}
\newtheorem{notation}[fact]{Notation}
\theoremstyle{remark}
\newtheorem{remark}[fact]{Remark}

\newcommand{\floor}[1]{\lfloor#1\rfloor}
\newcommand{\ceil}[1]{\lceil#1\rceil}
\newcommand{\rimpl}{\:\:\Longleftarrow\:\:}
\newcommand{\impll}{\:\:\Longrightarrow\:\:}
\newcommand{\impl}{\impll}%!!!{\:\:\Rightarrow\:\:}
\newcommand{\iffl}{\:\:\Longleftrightarrow\:\:}
\newcommand{\qimpl}{``$\Rightarrow$''}
\newcommand{\qrimpl}{``$\Leftarrow$''}

\newcommand{\parref}[1]{(\ref{#1})}
\newcommand{\by}[1]{{#1}}
\newcommand{\bydef}{\by{def}}
\newcommand{\bynot}{\by{not}}
\newcommand{\byhyp}{\by{hyp}}
\newcommand{\symby}[2]{\stackrel{#1}{{#2}}}
\newcommand{\symbyrm}[2]{\symby{{\rm\scriptstyle{#1}}}{#2}}
\newcommand{\symbyit}[2]{\symby{{\mit\scriptstyle{#1}}}{#2}}
\newcommand{\implby}[1]{\symbyrm{#1}{\impll}}
\newcommand{\implbyref}[1]{\implby{\parref{#1}}}
\newcommand{\implbyhyp}{\implby{\byhyp}}
\newcommand{\iffby}[1]{\symbyrm{#1}{\iffl}}
\newcommand{\iffbydef}{\iffby{\bydef}}
\newcommand{\iffbyref}[1]{\iffby{\parref{#1}}}
\newcommand{\eqby}[1]{\symbyrm{#1}{=}}
\newcommand{\eqbydef}{\eqby{\bydef}}
\newcommand{\eqbyref}[1]{\eqby{\parref{#1}}}
\newcommand{\eqbyhyp}{\eqby{\byhyp}}
\newcommand{\eqbynot}{\eqby{\bynot}}
\newcommand{\leby}[1]{\symbyrm{#1}{\le}}
\newcommand{\lebyref}[1]{\leby{\parref{#1}}}
\newcommand{\lebyhyp}{\leby{\byhyp}}
\newcommand{\geby}[1]{\symbyrm{#1}{\geq}}
\newcommand{\gebyref}[1]{\geby{\parref{#1}}}
\newcommand{\gebyhyp}{\geby{\byhyp}}
\newcommand{\gt}{>}
\newcommand{\gtby}[1]{\symbyrm{#1}{\gt}}
\newcommand{\gtbyref}[1]{\gtby{\parref{#1}}}
\newcommand{\gtbyhyp}{\gtby{\byhyp}}
\newcommand{\lt}{<}
\newcommand{\ltby}[1]{\symbyrm{#1}{\lt}}
\newcommand{\ltbyref}[1]{\ltby{\parref{#1}}}
\newcommand{\ltbyhyp}{\ltby{\byhyp}}
\newcommand{\inby}[1]{\symbyrm{#1}{\in}}
\newcommand{\inbydef}{\inby{\bydef}}
\newcommand{\inbyref}[1]{\inby{\parref{#1}}}
\newcommand{\inbyhyp}{\inby{\byhyp}}
\newcommand{\inbynot}{\inby{\bynot}}
\newcommand{\sseteqby}[1]{\symbyrm{#1}{\subseteq}}
\newcommand{\sseteqbydef}{\sseteqby{\bydef}}
\newcommand{\sseteqbyref}[1]{\sseteqby{\parref{#1}}}
\newcommand{\sseteqbyhyp}{\sseteqby{\byhyp}}
\newcommand{\sseteqbynot}{\sseteqby{\bynot}}

\newcommand{\set}[2]{\lc #1 \thickspace|\thickspace #2 \rc}
\newcommand{\minset}[2]{\min\set{#1}{#2}}
\newcommand{\txt}[1]{\:\text{#1}\:}

\newcommand{\mc}{,\:}
\newcommand{\ms}{\:}
\newcommand{\mcbs}{\mc\;\;}
\newcommand{\mcss}{\mc\;}

\newcommand{\lp}{\left(}
\newcommand{\lb}{\left[}
\newcommand{\lc}{\left\{}
\newcommand{\rp}{\right)}
\newcommand{\rb}{\right]}
\newcommand{\rc}{\right\}}
\newcommand{\la}{\left\langle}
\newcommand{\ra}{\right\rangle}

\newcommand\lnref[1]{line \lnref*{#1}}
\WithSuffix
\newcommand\lnref*[1]{\#\ref{#1}}
\newcommand\lnrefs[2]{lines \lnrefs*{#1}{#2}}
\WithSuffix
\newcommand\lnrefs*[2]{\lnref*{#1}\~\lnref*{#2}}

\newcommand{\factref}[1]{fact \ref{#1}}
\newcommand{\Factref}[1]{Fact \ref{#1}}

\newcommand\Nat{\mathbb{N}}
\WithSuffix
\newcommand\Nat*{\mathbb{N}^*}

\newcommand\Int{\mathbb{Z}}
\WithSuffix
\newcommand\Int*{\mathbb{Z}^*}

\newcommand\Real{\mathbb{R}}
\WithSuffix
\newcommand\Real*{\mathbb{R}^*}

\renewcommand{\contentsname}{Table of Contents}
\renewcommand{\cftsecleader}{\cftdotfill{\cftdotsep}}

\title{A String Minimum Deletetions Problem}
\author{\larger\vargyas\medskip\\ \tt\textbf{stvar@yahoo.com}}
\date{Jun 17, 2021}

% http://tex.stackexchange.com/questions/236781/align-the-footnote-markers-to-the-right-in-footnotes/236802#236802
\makeatletter
\patchcmd{\@makefntext}
    {\@makefnmark\hss}
    {\hss\@makefnmark\hspace{4pt}}
    {}{}
\makeatletter

\begin{document}
\maketitle
\tableofcontents

\section{The Problem Definition and Its Solution}

The string minimum deletions problem to which this note refers
to is defined as follows:

\begin{quoting}
Given a non-empty sequence of characters, where each character
is allowed to be one of the two letters $A$ or $B$, do find the
minimum number of character deletions that makes the given
sequence become of form $A^\ast\! B^\ast$.
\end{quoting}

The algorithm that solves this problem is very simple, as shown
by the following \cplusplus\ and \python\ implementations:

\medskip

\begin{cpplisting}
size_t min_del(const std::string& S)
{
    size_t m = 0, b = 0;
    for (auto c : S) {
        if (c == 'A')
            m = std::min(m + 1, b);
        else
            b ++;
    }
    return m;
}
\end{cpplisting}

\begin{pylisting}
def min_del(S)
    m, b = 0, 0
    for c in S:
        if c == 'A':
            m = min(m + 1, b)
        else:
            b += 1
    return m
\end{pylisting}

\break

\section{Mathematical Proof of the Algorithm's Correctness}

\begin{definition}
\begin{equation}
s\in\lc A, B\rc^\ast \txt{is {\it valid}} \iffbydef
\exists\ms n, m\in\Nat \txt{such that} s = A^n B^m. \label{defvalid}
\end{equation}
\end{definition}

\begin{notation}

\begin{align}
& E : \lc A, B \rc \ \times \lc 0, 1 \rc \to \lc \lambda, A, B \rc \label{defE} \\
%
& E \lp x, 0 \rp \eqbydef \lambda\mc 
  E \lp x, 1 \rp \eqbydef x \nonumber \\
%
&  E_n : \lc A, B \rc^n \ \times \lc 0, 1 \rc^n \to \lc A, B \rc^\ast \label{defEn} \\
%
&  E_n \lp s, e \rp \eqbydef E \lp s_1, e_1 \rp \cdots E \lp s_n, e_n \rp\mc\txt{for}
   s \in \lc A, B \rc^n\mc
   e = \la e_1, \cdots, e_n\ra \in \lc 0, 1 \rc^n \nonumber \\
%
& L_n : \lc 0, 1 \rc^n \to \Nat\mc \label{defLn} \\
%
& L_n \lp e \rp \eqbydef \neg e_1 + \cdots + \neg e_n\mc\txt{for}
  e = \la e_1, \cdots, e_n\ra \in \lc 0, 1 \rc^n \nonumber \\
%
& \mathcal{L}_n : \lc A, B \rc^n \to \mathcal{P}(\Nat) \label{defLLn}\\
%
& \mathcal{L}_n \lp s \rp \eqbydef \set{L_n\lp e\rp}{e \in \lc 0, 1\rc^n
    \txt{and} E_n \lp s, e \rp \txt{is valid}} \nonumber \\
%
& M_n : \lc A, B \rc^n \to \Nat \label{defMn} \\
%
& M_n \lp s \rp \eqbydef \min\mathcal{L}_n\lp s\rp \nonumber \\
%
& \sharp : \lc A, B \rc^\ast \times \lc A, B \rc \to \Nat \label{defsharp} \\
%
& \sharp\lp s, x\rp \eqbydef \txt{the number of occurrences of} x \txt{in} s \nonumber
\end{align}

\end{notation}

\begin{remark}
For $m \in\Nat$ and $\phi \ne X \in \mathcal{P}\lp\Nat\rp$,
$m = \min X$ if and only if:
\begin{align}
& m \in X \label{minin} \txt{and} \\
& m \le x, \txt{for all} x \in X \label{minle}
\end{align}
\end{remark}

\begin{proposition}
For all $X, Y \in \mathcal{P}\lp\Nat\rp$ and $a \in \Nat$, we have that:
%
\begin{align}
& \phi \ne X \subseteq Y \impl \min X \ge \min Y \label{minXgeY} \\
& \min X + \min Y = \min X + Y\mc \label{minXplusY}
  \txt{where} X + Y \eqbydef \set{x + y}{x \in X\mc y \in Y} \in \mathcal{P}\lp\Nat\rp \\
& a + \min X = \min \lp a + X\rp\mc \label{minaplusX}
\end{align}
\end{proposition}

\begin{proof}
For \parref{minXgeY}, we have that:
%
\begin{equation*}
\min X \inbyref{minin} X
    \implbyhyp \min X \in Y
    \implbyref{minle} \min X \ge \min Y
\end{equation*}
%
For \parref{minXplusY}, let $m \eqbynot \min X + \min Y$. On one hand
we have that \parref{minin} holds for $m$ and $X + Y$:
%
\begin{align*}
\min X \inbyref{minin} X \mc
\min Y \inbyref{minin} Y \impl
m = \min X + \min Y \in X + Y
\end{align*}
%
On the other hand, \parref{minle} holds for $m$ and $X + Y$:
\begin{align*}
z \in X + Y & \impl \exists\ms x \in X, y \in Y \txt{such that} z = x + y \\
            & \implbyref{minle} \min X \le x \mc \min Y \le y \\
            & \impl m = \min X + \min Y \le x + y = z
\end{align*}
\parref{minaplusX} is true by \parref{minXplusY} taking $Y = \lc a\rc$.
\end{proof}

\begin{proposition}
For all $s \in \lc A, B\rc^n$ and $x \in\lc A, B\rc$,
we have that:
%
\begin{align}
\mathcal{L}_n \lp s \rp & \subseteq \mathcal{L}_{n+1} \lp s B\rp
    \label{LLnsubseteqLLnplus1} \\
M_n \lp s \rp & \ge M_{n+1} \lp s B\rp
    \label{MngeMnplus1} \\
M_n \lp s \rp & \le M_{n+1} \lp s x\rp
    \label{MnleMnplus1}
\end{align}
\end{proposition}

\begin{proof}
For \parref{LLnsubseteqLLnplus1}, we have that, for an arbitrary $l$:
%
\begin{align*}
l \in \mathcal{L}_n \lp s \rp 
\implbyref{defLLn} & \exists\ms e \in \lc 0, 1 \rc^n \txt{such that}
    E_n \lp s\mc e \rp \txt{is valid and} L_n \lp e \rp = l \\
\impl & \exists\ms e'\eqbydef \la e_1\mc\cdots\mc e_n\mc 1\ra \in \lc 0, 1 \rc^{n+1} \txt{such that} \\
& \begin{array}{l}
    E_{n+1} \lp s B\mc e' \rp
      \eqbyref{defEn} E_n \lp s\mc e \rp E\lp B, e'_{n+1} \rp
      \eqbyref{defE} E_n \lp s\mc e \rp B \\
    L_{n+1} \lp e' \rp \eqbyref{defLn} L_n \lp e \rp + \neg e'_{n+1} = L_n \lp e\rp = l
  \end{array} \\
\impl & \exists\ms e'\in \lc 0, 1 \rc^{n+1} \txt{such that}
  E_{n+1} \lp s B\mc e' \rp \txt{is valid and}
  L_{n+1} \lp e' \rp = l \\
\implbyref{defLLn} & l \in \mathcal{L}_{n+1} \lp s B \rp
\end{align*}
%
\parref{MngeMnplus1} follows then from \parref{LLnsubseteqLLnplus1} and
\parref{minXgeY}.
\textpar
For \parref{MnleMnplus1}, let $x\in\lc A, B\rc$ be arbitrary; we have that:
%
\begin{align*}
m \eqbynot M_{n+1} \lp s x \rp 
\implbyref{minin} & m \in \mathcal{L}_{n+1} \lp s x \rp \\
\implbyref{defLLn} & \exists\ms e \in \lc 0, 1 \rc^{n+1} \txt{such that}
    E_{n+1} \lp s x\mc e \rp \txt{is valid and} L_{n+1} \lp e \rp = m \\
\impl & \exists\ms e'\eqbydef \la e_1\mc\cdots\mc e_n\ra \in \lc 0, 1 \rc^n \txt{such that} \\
& \begin{array}{l}
    E_{n+1} \lp s x\mc e \rp
      \eqbyref{defEn} E_n \lp s\mc e' \rp E\lp x, e_{n+1} \rp \\
    L_{n+1} \lp e \rp \eqbyref{defLn} L_n \lp e' \rp + \neg e_n \ge L_n \lp e'\rp
  \end{array} \\
\implbyref{defvalid} & \exists\ms e'\in \lc 0, 1 \rc^n \txt{such that}
  E_n \lp s\mc e' \rp \txt{is valid and}
  L_n \lp e' \rp \le m \\
\implbyref{defLLn} & \exists\ms l\in \mathcal{L}_n \lp s \rp
  \txt{such that} m \ge l \\
\implbyref{minle} & m \ge \min\mathcal{L}_n \lp s \rp
  \eqbyref{defMn} M_n \lp s \rp
\end{align*}
%
\end{proof}

\begin{proposition}
For all $s \in \lc A, B\rc^n$ and $x \in \lc A, B\rc$, we have that:
%
\begin{align}
\mathcal{L}_{n+1} \lp s x \rp & \supseteq \mathcal{L}_n \lp s\rp + 1
    \label{LLnsupseteqLLnplus1} \\
M_{n+1} \lp s x \rp & \le M_n \lp s \rp + 1
    \label{Mnplus1geMnplus1} \\
\mathcal{L}_{n+1}\lp s A\rp & \ni\sharp\lp s\mc B\rp
    \label{sharpin}\\
M_{n+1}\lp s A \rp & \le \sharp\lp s\mc B\rp
    \label{Mnplus1lesharp} \\
M_{n+1}\lp s A \rp & \ge M_n\lp s \rp + 1 \txt{or}
    \label{Mnplus1geMnorsharp} \\
M_{n+1}\lp s A \rp & \ge \sharp\lp s\mc B\rp \nonumber
\end{align}
\end{proposition}

\begin{proof}
For \parref{LLnsupseteqLLnplus1}, we have that, for an arbitrary $l$:
%
\begin{align*}
l \in \mathcal{L}_n \lp s \rp + 1
\implbyref{defLLn} & \exists\ms e \in \lc 0, 1 \rc^n \txt{such that}
    E_n \lp s\mc e \rp \txt{is valid and} l = L_n \lp e \rp + 1 \\
\impl & \exists\ms e'\eqbydef \la e_1\mc\cdots\mc e_n\mc 0\ra \in \lc 0, 1 \rc^{n+1} \txt{such that} \\
& \begin{array}{l}
    E_{n+1} \lp s x\mc e' \rp
      \eqbyref{defEn} E_n \lp s\mc e \rp E\lp x, e'_{n+1} \rp
      \eqbyref{defE} E_n \lp s\mc e \rp \\
    L_{n+1} \lp e' \rp \eqbyref{defLn} L_n \lp e \rp + \neg e'_{n+1} = L_n \lp e\rp + 1 = l
  \end{array} \\
\impl & \exists\ms e'\in \lc 0, 1 \rc^{n+1} \txt{such that}
  E_{n+1} \lp s x\mc e' \rp \txt{is valid and}
  L_{n+1} \lp e' \rp = l \\
\implbyref{defLLn} & l \in \mathcal{L}_{n+1} \lp s x \rp
\end{align*}
%
\parref{Mnplus1geMnplus1} follows then from \parref{LLnsupseteqLLnplus1} and
\parref{minXgeY}.
\textpar
For \parref{sharpin}, let have $e \in\lc 0, 1\rc^{n+1}$ such that it
deletes all $B$s from $s A$. Then, we have that:
%
\begin{align*}
L_{n+1}\lp e\rp = \sharp\lp s A\mc B\rp = \sharp\lp s\mc B\rp
\end{align*}
Consequently, by \parref{defLLn}, we obtain \parref{sharpin}.
\textpar
For \parref{Mnplus1lesharp}, apply \parref{sharpin}, \parref{minle} and \parref{defMn}. 
\textpar
For \parref{Mnplus1geMnorsharp}, taking $l$ arbitrarily, we have that:
%
\begin{align*}
l \in \mathcal{L}_{n+1}\lp s\rp
  \implbyref{defLLn} \exists\ms e\in\lc 0, 1\rc^{n+1} \txt{such that}
    & E_{n+1}\lp s A\mc e\rp \txt{is valid and} \\
    & L_{n+1}\lp e\rp = l
\end{align*}
%
Let $e' \eqbynot \la e_1\mc\cdots\mc e_n\ra\in\lc 0, 1\rc^n$.
In case $e_{n+1} = 0$:
%
\begin{align*}
& E_{n+1}\lp s A\mc e\rp \eqbyref{defEn} E_n\lp s\mc e'\rp\mc
  L_{n+1}\lp e\rp \eqbyref{defLn} L_n\lp e'\rp + \neg e_{n+1} = L_n\lp e'\rp + 1\\
& \implbyref{defLLn} l \in \mathcal{L}_n\lp s\rp + 1
  \implbyref{minle} l \ge \min\mathcal{L}_n\lp s\rp + 1 \eqbyref{defMn} M_n\lp s\rp + 1
\end{align*}
%
In case $e_{n+1} = 1$:
%
\begin{align*}
& E_{n+1}\lp s A\mc e\rp \eqbyref{defEn} E_n\lp s\mc e'\rp A\mc
  L_{n+1}\lp e\rp \eqbyref{defLn} L_n\lp e'\rp + \neg e_{n+1} = L_n\lp e'\rp \\
& \implbyref{defvalid} E_n\lp s\mc e'\rp \txt{does not contain} B\txt{s}
  \implbyref{defLn} l \ge \sharp\lp s\mc B\rp
\end{align*}
%
Consequently:
\begin{align*}
l \in \mathcal{L}_{n+1}\lp s\rp
  \impl l \ge M_n\lp s\rp + 1 \txt{or}
        l \ge \sharp\lp s\mc B\rp
\end{align*}
%
Due to the arbitrareness of $l$, by \parref{defMn}, the last implication above
proves \parref{Mnplus1geMnorsharp}.
%
\end{proof}

\begin{fact}
For all $s \in \lc A, B \rc^n$, we have that:
\begin{align}
M_{n+1}\lp s A\rp & = \min\lc\ms M_n\lp s \rp + 1\mc \sharp\lp s \mc B \rp\ms\rc
    \label{Mnplus1eqminMnplus1} \\
M_{n+1}\lp s B\rp & = M_n\lp s \rp
    \label{Mnplus1eqMn}
\end{align}
\end{fact}

\begin{proof}
%
For \parref{Mnplus1eqminMnplus1}, apply
\parref{Mnplus1geMnplus1},
\parref{Mnplus1lesharp} and
\parref{Mnplus1geMnorsharp}.
\textpar
For \parref{Mnplus1eqMn}, apply
\parref{MngeMnplus1} and
\parref{MnleMnplus1}.
%
\end{proof}

%
\let\thefootnote\relax%
\footnotetext{%
Algorithms Notes \copyright\ by \textbf{\vargyas}

Algorithms Notes is licensed under a
Creative Commons Attribution 4.0 Unported
License.

You should have received a copy of the
license along with this work.  If not, see
\url{http://creativecommons.org/licenses/by/4.0/}.}
%
\end{document}

